%% include header:
\input{./header}

%% include template:
\input{./templates/metropolis_cert}


%% Title:
%% ----------------------------------------

\title{Compboost}
\subtitle{Modular framework for component-wise boosting}
\date{\today}
\author{Daniel Schalk}
\institute{LMU Munich\\Working Group Computational Statistics}

%% Wrap Shaded around Shunk to have a nices R output:
%% --------------------------------------------------

% Include before begin document to have Schunk:
\usepackage{Sweave}

\let\OldSchunk\Schunk
\let\endOldSchunk\endSchunk

\renewenvironment{Schunk}
 {\small\begin{Shaded}\OldSchunk}
 {\endOldSchunk\end{Shaded}\normalsize}

%% Prevent code from printing over margin:
%% --------------------------------------------------
 

%% Content:
%% ----------------------------------------

\begin{document}
\Sconcordance{concordance:demo_slides.tex:demo_slides.Rnw:%
1 31 1 1 4 15 1}
\Sconcordance{concordance:demo_slides.tex:./chapters/intro.Rnw:ofs 48:%
1 10 1 1 2 8 0 1 2 8 1 1 5 1 2 11 1}
\Sconcordance{concordance:demo_slides.tex:demo_slides.Rnw:ofs 90:%
53 1 1}
\Sconcordance{concordance:demo_slides.tex:./chapters/latex_math.Rnw:ofs 92:%
1 19 1}
\Sconcordance{concordance:demo_slides.tex:demo_slides.Rnw:ofs 112:%
56 7 1}


\maketitle

\begin{frame}[plain]{Table of contents}
	\setbeamertemplate{section in toc}[sections numbered]
	\tableofcontents[hideallsubsections]
\end{frame}


\section{About the Template}
\begin{frame}[fragile]{Manual Code Chunks}

\begin{figure}
\centering
\includegraphics[width=0.7\textwidth]{images/comp_boosting.png}
\end{figure}

\end{frame}

\begin{frame}[fragile]{R Code Chunks}

\begin{Schunk}
\begin{Sinput}
> rnorm(10)
\end{Sinput}
\begin{Soutput}
 [1] -0.03864  0.23352 -0.76209  0.61107 -1.11515  1.25164
 [7]  0.35639  1.40238 -0.01200 -0.27966
\end{Soutput}
\end{Schunk}

\end{frame}

\begin{frame}[fragile]{R Plot Chunks}

To include a \alert{centered plot}, \texttt{Sweave} you need to wrap
\texttt{center} environment:

\begin{center}
\includegraphics{demo_slides-003}
\end{center}

\end{frame}

\begin{frame}{URLs}

URLs can be easily included by \texttt{\textbackslash url\{myurl\}} and are illustrated in
orange:
\begin{center}
  \url{https://mlr-org.github.io/mlr-tutorial/devel/html/}
\end{center}
\end{frame}

\section{Include latex-math}
\begin{frame}[fragile]{Setting Up Latex-Math}

All you need to do is to clone the \texttt{latex-math} repository into
\texttt{./latex\_pres}:
\begin{verbatim}
git clone http://www.github.com/compstat-lmu/latex-math
\end{verbatim}

\alert{Note:} If you do not have cloned the repo the code should also compile.
Anyway, you then are not able to use \texttt{latex-math}, obviously.

\end{frame}

\begin{frame}{Latex-Math in Action}

\[
\thetah = \argmin_{\theta \in \Theta} \sumin \Lxyi
\]

\end{frame}


\begin{frame}[plain, standout]
  Questions?
\end{frame}


\end{document}
